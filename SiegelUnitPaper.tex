\documentclass[10pt]{amsart}
\usepackage{tikz,array, verbatim}
\usepackage{amsfonts, amsmath, latexsym, epsfig, caption}
\usepackage{amssymb, color}
%\newcommand{\todo}[1]{{\color{red}#1}}
\usepackage{epsf}
%\usepackage{cite}
\usepackage{array}
\usepackage{ragged2e}
\usepackage{hyperref}
        \headheight=7pt
        \textheight=574pt
%        \textwidth=432pt
        \textwidth=400pt
        \topmargin=14pt
        \oddsidemargin=18pt
        \evensidemargin=18pt
%\DeclareMathOperator{\GL}{GL}
\DeclareMathOperator{\SL}{SL}
\DeclareMathOperator{\Stab}{Stab}
\DeclareMathOperator{\AGL}{AGL}
%\DeclareMathOperator{\Sym}{Sym}
\DeclareMathOperator{\relint}{relint}
\DeclareMathOperator{\tconv}{conv}
\DeclareMathOperator{\vol}{vol}
\DeclareMathOperator{\tcvp}{cvp}
\DeclareMathOperator{\tCVP}{CVP}
\DeclareMathOperator{\tvert}{vert}
\DeclareMathOperator{\Del}{Del}
\DeclareMathOperator{\parity}{Parity}
\include{Includes/operators}
\include{Includes/fonts}

\title{Einheiten Quadratischer Formen: Units of Quadratic Forms}

\def\QuotS#1#2{\leavevmode\kern-.0em\raise.2ex\hbox{$#1$}\kern-.1em/\kern-.1em\lower.25ex\hbox{$#2$}}


%\usepackage{vmargin}
%\setpapersize{custom}{21cm}{29.7cm}
%\setmarginsrb{1.7cm}{1cm}{1.7cm}{3.5cm}{0pt}{0pt}{0pt}{0pt}
%marge gauche, marge haut, marge droite, marge bas.
\urlstyle{sf}

\newcommand\ma[1]{{\color{purple}(Mario: #1)}}
\newcommand\mk[1]{{\color{purple}#1}}

\begin{document}

\subjclass[2010]{Primary: 11H55, 52C07. Keywords: Iso-edge domains, Conway--Sloane conjecture, toroidal compactification}

\author[C.L. Siegel]{Carl Ludwig Siegel}

\author[M. D. Sikiri\'c]{Mathieu Dutour Sikiri\'c}
\address{Mathieu Dutour Sikiri\'c, Rudjer Boskovi\'c Institute, Bijeni\v{c}ka 54, 10000 Zagreb, Croatia}
\email{mathieu.dutour@gmail.com}

\author[N. Bogachev]{Nikolay Bogachev}
\address{University of Toronto, Canada}
\email{nvbogach@mail.ru}


\newcommand{\RR}{\ensuremath{\mathbb{R}}}
\newcommand{\NN}{\ensuremath{\mathbb{N}}}
\newcommand{\QQ}{\ensuremath{\mathbb{Q}}}
\newcommand{\CC}{\ensuremath{\mathbb{C}}}
\newcommand{\ZZ}{\ensuremath{\mathbb{Z}}}
\newcommand{\TT}{\ensuremath{\mathbb{T}}}

\newtheorem{theorem}{Theorem}[section]
\newtheorem{proposition}[theorem]{Proposition}
\newtheorem{corollary}[theorem]{Corollary}
\newtheorem{lemma}[theorem]{Lemma}
\newtheorem{problem}[theorem]{Problem}
\newtheorem{conjecture}{Conjecture}
\newtheorem{question}{Question}
\newtheorem{claim}{Claim}
\newtheorem{remark}[theorem]{Remark}
\theoremstyle{definition}
\newtheorem{definition}[theorem]{Definition}

\maketitle

Under the units of a quadratic form $x^T \Sigma x = \sum{k,l=1}^m s_{kl} x_k x_l$ with
rational coefficient $s_{kl}$ we understand the matrices of those integral transformations
$x\mapsto Cx$ of the variables $x_1$, \dots, $x_m$ which leave this quadratic form invariant,
i.e. satisfy the matrix equation $C^T \Sigma C = \Sigma$.
If the determinant $det \Sigma \not= 0$, then
$det C = \pm 1$; thus $C = -1$ is also an integral matrix and the units then form a group
under multiplication, the group of units of $\Sigma$. While for definite quadratic forms
the unit group is always of finite order, it turns out that indefinite quadratic forms,
with a trivial exception, always have infinitely many units.
In the case of $m = 2$, the determination of units leads to the solution of the
Pell equation $t^2 - D u^2 = \pm 4$, where $D = det \Sigma$. The method given by Lagrange
for solving this equation also provides all units in the binary case. The theory
of units of ternary indefinite quadratic forms was treated by Hermite [1], using a
far-reaching idea that can also be used for arbitrary number of variables. However,
Hermite’s investigations into the theory of indefinite quadratic forms left a gap that
was only filled by a paper [2] by Stouff.
Due to the importance of unit groups for the arithmetic of quadratic forms, for
the theory of functions, and for general group theory, a brief coherent introduction
to the theory of units may be desirable, as will be given in the following.

\section{Fundamental domains of discontinuous groups}

Let $G$ be an open domain in $h$-dimensional real Euclidean space, and let $\Gamma$ be a
countable group of unique mappings of this domain onto itself. The group is said
to be discontinuous in $G$ if the sequence of image points of an arbitrary point of $G$
has no accumulation point within $G$. A subregion $F$ of $G$ shall be a fundamental
domain of $\Gamma$ (with respect to $G$) if every point of $G$ either has exactly one image
point within the interior of $F$, or has at least one image point on the boundary
of $F$. An important task is to find a fundamental domain (as simple as possible)
for a given discontinuous group. While solving this problem, it is possible to gain
deeper insights into the properties of the group, for instance, in certain cases, the
construct of the group from only a finite set of generators.

The problem of proving the discontinuity of a group and finding a suitable fundamental
domain can be modified as follows. Let G be uniquely mapped onto a domain $G^*$ of Euclidean
space of $h^*$ dimensions, where $h^*$ can also be less than $h$.
Then, each mapping of the group $\Gamma$ corresponds to a mapping of $G^*$ onto itself,
which, however, may not be unique. We now assume that these mappings from
$G^*$ onto itself are all unique, i.e., that any two points $P$ and $Q$ of $G$, which are
associated with the same point $P^* = Q^*$ of $G^*$, are transformed by each mapping of
$\Gamma$ into two points $P_1$ and $Q_1$ of $G$, to which the same point $P_1^* = Q_1^*$
of $G^*$ is again assigned. Then, a group $\Gamma^*$ of unique mappings of the domain
$G^*$ onto itself arises again. All mappings from $\Gamma$, to which the identical mapping
from $\Gamma^*$ is assigned, form an invariant subgroup $\Gamma_0$ of $\Gamma$, and $\Gamma^*$
is the factor group $\Gamma / \Gamma_0$. Our task can now be divided into possibly simpler
parts: first, determine a fundamental domain $F^*$ for $\Gamma^*$ with respect to $G^*$,
then the subregion $G_0$ of $G$, which corresponds to $F^*$ in $G^*$, and finally, a
fundamental domain $F$ for $\Gamma_0$ with respect to $G_0$. Then $F$
is also a fundamental domain of $\Gamma$ with respect to $G$.

We now consider specifically the group of unimodular substitutions $x \mapsto Ux$ of $m$
real variables $x_1$, \dots, $x_m$ are combined into a column vector. Here, all $m$-row
matrices $U$ with determinant $\pm 1$ and integer elements are allowed. For $m > 1$, this
group is not discontinuous in any domain of the $m$-dimensional Euclidean space, as
it can be easily seen that every point $x$ with rational coordinates is a fixed point
of infinitely many matrices $U$. In order to move to a higher dimensional space in
which the group is discontinuous, instead of a single column, we take $m$ independent
columns $x_1$, \dots, $x_m$, and thus have a matrix $X$ with m rows and columns. The group
$\Gamma$ of mappings $X \mapsto UX$ of the domain $G$ of the ($h = m^2$)-dimensional
space defined by the condition $det X\not= 0$ is discontinuous; for if a sequence $U_k X$ ($k = 1, 2, \dots$)
of unimodular matrices converges, then as a contradiction, the sequence $U_k$ itself
would converge.

To determine a fundamental domain $F$ of $\Gamma$ with respect to $G$, Minkowski [3]
examined the domain $G^*$ of only $h^* = m(m + 1)/2$ (rounded up) dimensions,
which is given by the coefficients $h_{kl}$ ($1 \leq k \leq m$) of the symmetric matrix
$\Phi = X X^T = (h_{kl})$. Thus, $G^*$ is the space of coefficients of positive
definite quadratic forms in $m$ variables. The mapping $X \mapsto UX$ of the $G$-space
corresponds to the mapping $\Phi \mapsto U \Phi U^T$ of the $\Phi$-space, and this
is the identity mapping only for $\pm I$, where $I$ is the identity matrix.
Therefore, the group $\Gamma^*$ of mappings of the $\Phi$-space to be considered
is precisely the factor group of the group $\Gamma$ in respect to the invariant
subgroup $\Gamma_0$ generated by $I$ and $-I$, and the elements of $\Gamma^*$ are
obtained from those of $\Gamma$ by not considering $U$ and $-U$ as different.
For simplicity, we write $U^t \Phi U = \Phi[U]$ and say that $\Phi[U]$ is equivalent to $\Phi$).
Thus, determining a fundamental domain for the unimodular group with respect to the $X$-space reduces
to the appropriate selection of a representative from each class of equivalent positive
definite quadratic forms in m variables, which is the domain of reduction theory of
definite quadratic forms.

For $m = 2$, the reduction theory was developed by Lagrange, for $m = 3$ by Seeber, and for
arbitrary $m$ it was started by Hermite and completed by Minkowski.
More recently, Bieberbach and J. S. Chur have given [4] a simplified representation
of Minkowski’s reduction theory in a jointly written paper. As Minkowski showed,
a convex corner bounded by finitely many hyperplanes can be chosen as the fundamental domain $F^*$ of $\Gamma^*$
in the $\Phi$-space, and only finitely many image regions are adjacent to $F^*$.
We will first derive these two important results in a somewhat
simplified way and then generalize the second of them so that it can be used for
further applications

After the preparatory considerations for the theory of positive definite quadratic
forms, we now turn to the actual goal of our investigation, namely, the unit theory
of indefinite quadratic forms $x^t\Sigma x = \Sigma[x]$ with $m$ variables and rational
coefficients.
Let $(n, m-n)$ be the signature of $\Sigma$, i.e., let $\Sigma[x]$ be transformable by
a real invertible linear substitution of variables into the form $y_1^2 + \dots + y_n^2 - y_{n+1}^2 - \dots - y_m^2$.
We the consider the matrix equation
\begin{equation*}
\Phi \Sigma^{-1} \Phi = \Sigma
\end{equation*}
and require that $\Phi$ forms the matrix of a positive quadratic form. It turns out that
these $\Phi$ yield a manifold $H$ of exactly $n(m - n)$ dimensions. If $C$ is a unit of $\Sigma$,
i.e., $\Sigma[C] = \Sigma$, then $\Sigma^{-1} = \Sigma^{-1}[C]$ and it follows from (1) that
\begin{equation*}
\Phi[C] \Sigma^{-1} \Phi[C] = \Sigma.
\end{equation*}
Thus, $H$ is invariant under the mappings $\Phi \mapsto \Phi[C]$. It will be shown that
the unit group of $\Sigma$ is discontinuous in the $n(m - n)$-dimensional domain $H$ and
possesses a fundamental domain bounded by finitely many algebraic surfaces. At
the same time, a system of finitely many generators of the unit group is found.

The reduction theory of indefinite quadratic forms $\Sigma[x]$ has already been treated
by Hermite through the introduction of the positive quadratic form $\Phi[x]$ defined by
(1). We choose a unimodular matrix $U$ so that $\Phi[U]$ is located in the fundamental
domain $F^*$, i.e., in the domain of reduced positive quadratic forms, and then also
call $\Sigma[U]$ reduced. For integral $\Sigma$, Hermite stated the important theorem that
the elements of every reduced $\Sigma[U]$ lie between bounds that depend only on the
determinant of $\Sigma$. This theorem provides a tool for the study of the unit group of
$\Sigma$. As the proof given by Stouff is burdened by lengthy and opaque calculations,
the proof presented below may be more accessible.


\section{Reduction of positive forms}

A real symmetric matrix $S$ is positive, written as $S > 0$ if the quadratic form $S[x] > 0$
is for all $x\not= 0$. Then if $\mu$ is the minimum of $S[x]$ over the sphere $x^T x = 1$
we have $\mu > 0$ and because of the homogeneity we have
\begin{equation*}
S[r] \geq \mu r^t r.
\end{equation*}
From this it follows that $S[r]$ is ubounded on infinite sequence going to infinity.
In particular the function $S[r]$ achieves the minimum on any such sequences.

In the class of all matrices $S[U]$ equivalent to $S$ there is a Minkowski reduced one
determined by extremal condition. The first vector $u_1$ is obtained by selecting a vector
of minimal length $S[x]$. Then consider all unimodular matrices having $u_1$ as first
column and choose the second column so that its norm is as small as possible. Changing
sign if needed, we can make sure that $u_1 S u_2 \geq 0$. We can then do the third
column and normalize it with $u_2 S u_3\geq 0$. We continue until the $m$-th column is
built and so we have an unimodular matrix $U$ such that $S[U] = R$ which is then reduced.







\bibliographystyle{amsplain_initials_eprint}
\bibliography{RefIsoEdge}

\end{document}
